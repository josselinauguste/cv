%-------------------------------------------------------------------------------
%	SECTION TITLE
%-------------------------------------------------------------------------------
\cvsection{Expérience}


%-------------------------------------------------------------------------------
%	CONTENT
%-------------------------------------------------------------------------------
\begin{cventries}

  \cventry
    {Co-organisateur}
    {Organisation de conférences}
    {Bordeaux, France}
    {Depuis 2014}
    {
      \begin{cvsubentries}
        \cvsubentry{}{NewCrafts : conférence internationale axée sur l'amélioration des pratiques et la qualité du code}{2017, 2018}{}
        \cvsubentry{}{FableConf : la conférence européenne annuelle dédiée au langage F\# et au compilateur Javascript Fable}{2017}{}
        \cvsubentry{}{Agile Tour Bordeaux : événement annuel dédié à l’agilité réunissant 300 personnes sur 2 jours}{2016, 2017}{}
        \cvsubentry{}{Bdx I/O : conférence annuelle sur les technologies de développement réunissant 500 développeurs }{2014, 2015, 2016}{}
      \end{cvsubentries}
    }

  \cventry
    {\href{https://github.com/josselinauguste/my-talks}{https://github.com/josselinauguste/my-talks}}
    {Orateur}
    {}
    {}
    {}

  \cventry
    {Développeur expérimenté}
    {Bender Labs, P00LS{\enskip\cdotp\enskip}Finance décentralisée}
    {Bordeaux, France}
    {Mar. 2022 - Nov. 2025}
    {
    }

  \cventry
    {Tech Lead}
    {Sinch{\enskip\cdotp\enskip}Télécommunications}
    {Bordeaux, France}
    {Mar. 2020 - Mar. 2022}
    {
      \begin{cvitems}
        \item {Projets data analytics/facturation/connecteurs externes, puis gestion d'audience (équipes de 4-5 personnes).}
        \item {Membre des équipes de support pour le recrutement, la formation et la protection des données.}
        \item {Participation active à l'amélioration continue des pratiques organisationnelles et de développement, effort particulier sur l'adoption de DDD et les réflexions autour du découplage par domaine métier.}
        \item {Environnement : DDD, écosystème Python, MongoDB, Gitlab CI, AWS, Terraform, Packer, Holacracy, Feature Teams.}
      \end{cvitems}
    }

  \cventry
    {Développeur expérimenté}
    {Yaal{\enskip\cdotp\enskip}Investissement technique}
    {Bordeaux, France}
    {Oct. 2018 - Fév. 2020}
    {
      \begin{cvitems}
        \item {Développement back-end sur le projet myElefant, plateforme de marketing multi-canaux -- SMS, Messenger, RCS.}
        \item {Participation active à l'amélioration continue des pratiques organisationnelles et de développement.}
        \item {Environnement techique : écosystème Python, base de données objet ZODB, Mercurial, Ansible.}
      \end{cvitems}
    }

  \cventry
    {Responsable technique, associé gérant}
    {MesVaccins.net, Syadem{\enskip\cdotp\enskip}E-santé}
    {Bordeaux, France}
    {Oct. 2012 - Oct. 2018}
    {
      \begin{cvitems}
        \item {Encadrement de l'équipe technique : suivi des développements, consolidation de la vision technique, facilitation sur les pratiques et la culture de l'équipe.}
        \item {Stratégie produit : définition de la roadmap des produits, conciliation des besoins remontés par les utilisateurs avec la vision projet.}
        \item {Suivi des partenariats techniques : interface avec les équipes techniques et fonctionnelles des éditeurs intégrant nos services.}
        \item {Développement sur l'ensemble des applications conçues au sein du projet.}
        \item {Gestion administrative courante de l'entreprise et de ses employés.}
        \item {Environnement technique : Ruby et Ruby on Rails, Scala, F\#, PHP, Redis, React et React Native, CoffeeScript, Ansible, Docker.}
      \end{cvitems}
    }

  \cventry
    {Expert technique .Net}
    {Cdiscount{\enskip\cdotp\enskip}E-commerce}
    {Bordeaux, France}
    {Août 2010 - Sept. 2012}
    {
      \begin{cvitems}
        \item {Gestion technique de l'externalisation de projets en mode forfait.}
        \item {Industrialisation de la production.}
        \item {Validation des spécifications fonctionnelles et techniques.}
        \item {Pilotage technique des développements d'une équipe de 5 à 10 personnes.}
        \item {Développements sur les applications Front Office, Middle Office, et sur la couche de services et d'accès aux données.}
        \item {Environnement technique : Microsoft .Net, ASP.NET MVC, Silverlight 4, Windows Communication Foundation, SQL Server 2008, Transact-SQL.}
      \end{cvitems}
    }

  \cventry
    {Développeur back-end}
    {Business Anywhere{\enskip\cdotp\enskip}Édition de logiciels}
    {Bordeaux, France}
    {Juin 2010 - Juill. 2010}
    {
      \begin{cvitems}
        \item {Développement en forfait de deux projets pour le compte de grandes sociétés françaises souhaitant étoffer leurs services au travers d'applications mobiles. Périmètre : Back Office, Middle Office, web services REST.}
        \item {Environnement technique : Ruby on Rails, REST, Javascript, technologies web.}
      \end{cvitems}
    }

  \cventry
    {Développeur front-end}
    {Orange Vallée{\enskip\cdotp\enskip}Édition de logiciels}
    {Bordeaux, France}
    {Avr. 2010 - Juin 2010}
    {
      \begin{cvitems}
        \item {Développement d'un prototype technique pour la plateforme de diffusion audio Wormee éditée par Orange.}
        \item {Emphase sur les performances, la plateforme devant pouvoir répondre à un très fort trafic.}
        \item {Environnement technique : GWT 2, Java SE, Javascript, technologies web.}
      \end{cvitems}
    }

  \cventry
    {Ingénieur Études et Développements .Net}
    {Voxeet{\enskip\cdotp\enskip}Édition de logiciels}
    {Bordeaux, France}
    {Nov. 2009, Avril 2010}
    {
      \begin{cvitems}
        \item {Conception et développement de l'interface utilisateur d'un logiciel de conférence audio spatialisée.}
        \item {Recherche et développement sur l'expérience utilisateur et l'ergonomie du logiciel.}
        \item {Développement d'un protocole de transfert et des couches de capture/restitution de la voix dans un environnement spatialisé.}
        \item {Environnement technique : Microsoft .Net 3.5, Windows Presentation Foundation, protocole XMPP (openfire), codec Speex, OpenAL.}
      \end{cvitems}
    }

  \cventry
    {Ingénieur Études et Développements .Net}
    {Hager Group{\enskip\cdotp\enskip}Gestion d'énergie \& domotique}
    {Bordeaux, France}
    {Mai 2008 - Nov. 2009}
    {
      \begin{cvitems}
        \item {Conception et implémentation d'un logiciel de pilotage pour les installations domotiques, adapté aux interfaces tactiles et aux fortes disparités de matériel cible (tailles d'écran variables, terminaux à puissance de calcul limitée, etc.).}
        \item {Maquettage, étude ergonomique, recherche et développement sur la technologie,  implémentation, optimisations.}
        \item {Environnement technique : Microsoft .Net 3.5, Windows Presentation Foundation.}
      \end{cvitems}
    }

  \cventry
    {Ingénieur Études et Développements}
    {Innovantic{\enskip\cdotp\enskip}Édition de logiciel}
    {Bordeaux, France}
    {Mars 2007 - Août 2011}
    {
      \begin{cvitems}
        \item {Conception et développement d'une solution de sauvegarde en ligne : Tweam WebDisk. Développement back-end (serveur applicatif, base de données, stockage des données) et front-end (client lourd et client riche).}
        \item {Développement d'une solution de synchronisation de données entre terminaux mobiles et serveur d'application. Intégration de cette technologie au produit de bureau mobile Tweam MobileOffice. Étude de faisabilité, modélisation, prototypage, développement, mise au point et déploiement.}
        \item {Maintenance applicative et serveur sur Tweam MobileOffice.}
        \item {Environnement technique : Java SE 6, SWT, Hibernate, AJAX, Java ME, Windows Mobile, Blackberry, XML, UML.}
      \end{cvitems}
    }

  \cventry
    {Développeur de jeux vidéo}
    {BeTomorrow{\enskip\cdotp\enskip}Édition de solutions mobiles}
    {Bordeaux, France}
    {Juin - Sept. 2005, Juin - Sept. 2006}
    {
      \begin{cvitems}
        \item {Conception d'un moteur de jeu 3D pour les terminaux Java ME et Brew pour le compte de Lagardère Interactive. Recherche et développement sur les techniques de rendu logiciel. }
        \item {Portage de jeux existants sur l'ensemble de la gamme de téléphones de SFR. Mise en place de procédures de tests.}
        \item {Intégration au développement d'une gamme de jeux en ligne éditée par BeTomorrow.}
      \end{cvitems}
    }

%---------------------------------------------------------
\end{cventries}
